\chapter{Brainfuck}
\label{chap:ch2}

\par Brainfuck (commonly abbreviated as BF in academic literature) is an esoteric programming language created in 1993 by Swiss computer science student Urban Müller. It was conceived as a minimalist language that challenges conventional programming approaches by reducing syntactic elements to the bare essentials. The language's primary intent is not practical software development, but rather to serve as a vehicle for exploring computational theory, language design, and the boundaries of human-readable code.

\par Brainfuck operates on a simple model of computation: a one-dimensional array (commonly 30,000 cells in most implementations) of bytes initialized to zero, and a single data pointer that traverses this memory. Loops in Brainfuck are defined by matching \texttt{[} and \texttt{]} brackets, executing the enclosed code as long as the current cell is non-zero. The language lacks named variables, functions, or high-level abstractions, requiring the programmer to construct all control flow and data manipulation manually.

\par Despite its extreme simplicity, Brainfuck is Turing-complete. This implies that, in theory, any computable function can be implemented using Brainfuck, provided unlimited memory and time. The Turing-completeness of Brainfuck underscores its value as a pedagogical tool, illustrating how a minimal set of operations can express arbitrarily complex computation.

\par The language's deliberately obfuscated syntax forces programmers to think at the level of memory and instruction cycles, drawing parallels with low-level systems programming, particularly in assembly or machine code. This requirement to engage directly with memory manipulation and flow control makes Brainfuck a unique platform for studying program optimization, instruction translation, and interpreter design.

\par In contemporary settings, Brainfuck is primarily used for educational and experimental purposes. It remains a popular subject in programming contests, obfuscation challenges, and academic discourse within the esoteric programming community (esolangs). Its design exemplifies the power of minimalism in programming language theory and continues to inspire discussions on language expressiveness, compiler construction, and the boundaries of syntactic design.

\section{Machine specification}
\label{sec:ch2sec1}

\par The language is defined by an extremely compact instruction set consisting of only eight commands:

\begin{table}[H]
\centering
\begin{tabularx}{\textwidth}{| c | Y |}
\hline
+ & Increments the value at the current position that the machine points to. \\ \hline
- & Decrements the value at the current position that the machine points to. \\ \hline
\textless & Moves the pointer one cell to the left.                                  \\ \hline
\textgreater & Moves the pointer one cell to the right.                                 \\ \hline
[ & Jumps after the corresponding closed bracket when the value at the current cell is 0. \\ \hline
] & Jumps after the corresponding open bracket when the value at the current cell is not 0.   \\ \hline
. & Outputs the value at the current cell. \\ \hline
, & Read a value to be placed at the current cell. \\ \hline
\end{tabularx}
\caption{Brainfuck commands and their descriptions}
\end{table}

\subsection{Programming in the language}
\label{sec:ch2sec1sub1}

\par To facilitate a practical understanding of the Brainfuck language and its idiomatic constructs, we examine a canonical program that outputs the string 'Hello World!' to the screen:

\begin{minted}[
    linenos,                % add line numbers
    fontfamily=tt,          % typewriter font
    fontsize=\small,        % size
    breaklines,             % allow line breaks
    tabsize=4               % size of tab
]{brainfuck}
>++++++++[<+++++++++>-]<.
>++++[<+++++++>-]<+.
+++++++..+++.
>>++++++[<+++++++>-]<++.
------------.
>++++++[<+++++++++>-]<+.
<.
+++.
------.
--------.
>>>++++[<++++++++>-]<+.
\end{minted}

\par For clarity, the code has been formatted so that each line corresponds to the generation of a single character in the output. This organization aids in demonstrating how Brainfuck instructions directly map to output generation.

\par Consider the first line:

\begin{minted}[
    linenos,                % add line numbers
    fontfamily=tt,          % typewriter font
    fontsize=\small,        % size
    breaklines,             % allow line breaks
    tabsize=4               % size of tab
]{brainfuck}
>++++++++[<+++++++++>-]<.
\end{minted}

\par This sequence first increments the second memory cell (cell \#1) to the value 8. It then enters a loop that decrements cell \#1 on each iteration and simultaneously increments the third cell (cell \#2) by 9 per iteration. Once cell \#1 reaches zero, the loop terminates, and cell \#2 holds the value 72, corresponding to the ASCII code for 'H'. The final instruction outputs this value.

\par Similar patterns are applied across subsequent lines to construct the remaining characters of the phrase. Characters such as 'l', which appear multiple times, are output using consecutive . commands. The 'o' from “Hello” is reused in “World,” demonstrating efficient reuse of computed values.

\par This example illustrates common patterns and idioms in Brainfuck programming. One foundational construct is the multiplication loop:

\begin{minted}[
    linenos,                % add line numbers
    fontfamily=tt,          % typewriter font
    fontsize=\small,        % size
    breaklines,             % allow line breaks
    tabsize=4               % size of tab
]{brainfuck}
(any number of +)[(movement)(any number of +)(reverse movement)-]
\end{minted}

\par This structure multiplies the initial value of a cell by the number of additions performed inside the loop, effectively replicating multiplication via repeated addition.

\par Another fundamental idiom is the cell-clearing pattern, which resets the current cell's value to zero:

\begin{minted}[
    linenos,                % add line numbers
    fontfamily=tt,          % typewriter font
    fontsize=\small,        % size
    breaklines,             % allow line breaks
    tabsize=4               % size of tab
]{brainfuck}
[-]
\end{minted}

\par Additionally, a commonly used construct involves transferring the value of one cell to another:

\begin{minted}[
    linenos,                % add line numbers
    fontfamily=tt,          % typewriter font
    fontsize=\small,        % size
    breaklines,             % allow line breaks
    tabsize=4               % size of tab
]{brainfuck}
[(pointer movement)+(reverse pointer movement)-]
\end{minted}

\par This operation assumes that the destination cell is initially zero. If this is not the case, the original value will be added rather than overwritten. To enforce a pure move, the destination cell should be cleared first using the aforementioned clearing pattern.

\par These patterns provide valuable insights into the low-level logic of Brainfuck programs and are instrumental in designing interpreters or optimizing compilers targeting esoteric or constrained instruction sets.

\clearpage

\section{Emulator implementation}
\label{sec:ch2sec2}

\par Because of the language's simplicity, there have been many emulators made for it in all kinds of programming languages. As such, the focus will be put on the techniques utilised in the creation of optimized emulators rather than the emulators themselves, as they have been a rather exhausted subject.


\subsection{Emulating yourself?}
\label{sec:ch2sec2sub1}

\par 

\subsection{Simple interpreter}
\label{sec:ch2sec2sub2}

\par A basic Brainfuck interpreter operates by sequentially reading each character of the source code and executing the corresponding operation against a simulated memory model. The interpreter typically maintains a data array—commonly 30,000 bytes in size—initialized to zero, along with a data pointer that tracks the current cell. As the interpreter parses the code, it translates each command into its respective behavior: incrementing or decrementing the cell's value (+ or -), moving the data pointer left or right (< or >), reading input (,), or writing output (.). Control flow constructs ([ and ]) are implemented by jumping to the corresponding matching bracket based on the current cell's value. If the cell is zero at [, execution jumps forward to the instruction following the matching ]; otherwise, it continues normally. Similarly, ] causes a backward jump to the matching [ if the current cell is non-zero. To efficiently support these jumps, the interpreter often preprocesses the source code to map matching bracket pairs. The simplicity of this execution model makes Brainfuck interpreters ideal for illustrating core principles of language interpretation, including parsing, memory management, and flow control.

\subsection{Static compilation}
\label{sec:ch2sec2sub3}

\par Lorem ipsum dolor sit amet, consectetur adipiscing elit, sed do eiusmod tempor incididunt ut labore et dolore magna aliqua. Ut enim ad minim veniam, quis nostrud exercitation ullamco laboris nisi ut aliquip ex ea commodo consequat. Duis aute irure dolor in reprehenderit in voluptate velit esse cillum dolore eu fugiat nulla pariatur. Excepteur sint occaecat cupidatat non proident, sunt in culpa qui officia deserunt mollit anim id est laborum

\section{Applying optimizations}
\label{sec:ch2sec3}

\par Lorem ipsum dolor sit amet, consectetur adipiscing elit, sed do eiusmod tempor incididunt ut labore et dolore magna aliqua. Ut enim ad minim veniam, quis nostrud exercitation ullamco laboris nisi ut aliquip ex ea commodo consequat. Duis aute irure dolor in reprehenderit in voluptate velit esse cillum dolore eu fugiat nulla pariatur. Excepteur sint occaecat cupidatat non proident, sunt in culpa qui officia deserunt mollit anim id est laborum

\subsection{Precalculating jumps}
\label{subsec:ch2sec3sec1}

\par Lorem ipsum dolor sit amet, consectetur adipiscing elit, sed do eiusmod tempor incididunt ut labore et dolore magna aliqua. Ut enim ad minim veniam, quis nostrud exercitation ullamco laboris nisi ut aliquip ex ea commodo consequat. Duis aute irure dolor in reprehenderit in voluptate velit esse cillum dolore eu fugiat nulla pariatur. Excepteur sint occaecat cupidatat non proident, sunt in culpa qui officia deserunt mollit anim id est laborum

\subsection{Instruction folding}
\label{subsec:ch2sec3sec2}

\par Lorem ipsum dolor sit amet, consectetur adipiscing elit, sed do eiusmod tempor incididunt ut labore et dolore magna aliqua. Ut enim ad minim veniam, quis nostrud exercitation ullamco laboris nisi ut aliquip ex ea commodo consequat. Duis aute irure dolor in reprehenderit in voluptate velit esse cillum dolore eu fugiat nulla pariatur. Excepteur sint occaecat cupidatat non proident, sunt in culpa qui officia deserunt mollit anim id est laborum

\subsection{Sequence matching for common patterns}
\label{subsec:ch2sec3sec3}

\par Lorem ipsum dolor sit amet, consectetur adipiscing elit, sed do eiusmod tempor incididunt ut labore et dolore magna aliqua. Ut enim ad minim veniam, quis nostrud exercitation ullamco laboris nisi ut aliquip ex ea commodo consequat. Duis aute irure dolor in reprehenderit in voluptate velit esse cillum dolore eu fugiat nulla pariatur. Excepteur sint occaecat cupidatat non proident, sunt in culpa qui officia deserunt mollit anim id est laborum

\section{Testing}
\label{sec:ch2sec4}

\par 