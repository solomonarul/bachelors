\clearpage

\chapter{Brainfuck}
\label{chap:ch2}

\par Brainfuck (commonly abbreviated as BF in academic literature) is an esoteric programming language created in 1993 by Swiss computer science student Urban Müller\\\cite{BFWiki}. It was conceived as a minimalist language that challenges conventional programming approaches by reducing syntactic elements to the bare essentials. The language's primary intent is not practical software development, but rather to serve as a vehicle for exploring computational theory, language design, and the boundaries of human-readable code.

\par Brainfuck operates on a simple model of computation: a one-dimensional array of values initialized to zero, and a single data pointer that traverses this memory. Loops in Brainfuck are defined by matching \texttt{[} and \texttt{]} brackets, executing the enclosed code as long as the current cell is non-zero. The language lacks named variables, functions, or high-level abstractions, requiring the programmer to construct all control flow and data manipulation manually.

\par Despite its extreme simplicity, Brainfuck is Turing-complete. This implies that, in theory, any computable function can be implemented using Brainfuck, provided unlimited memory and time. The Turing-completeness of Brainfuck underscores its value as a pedagogical tool, illustrating how a minimal set of operations can express any arbitrarily complex computation.

\par The language's deliberately obfuscated syntax forces programmers to think at the level of memory and instruction cycles, drawing parallels with low-level systems programming, particularly in assembly or machine code. This requirement to engage directly with memory manipulation and flow control makes Brainfuck a unique platform for studying program optimization, instruction translation, and interpreter design.

\par In contemporary settings, Brainfuck is primarily used for educational and experimental purposes. It remains a popular subject in programming contests, obfuscation challenges, and academic discourse within the esoteric programming language (esolang) community. Its design exemplifies the power of minimalism in programming language theory and continues to inspire discussions on language expressiveness, compiler construction, and the boundaries of syntactic design.

\section{Machine specification}
\label{sec:ch2sec1}

\par The language is defined by an extremely compact instruction set consisting of only eight commands:

\begin{table}[H]
\centering
\begin{tabularx}{\textwidth}{| c | Y |}
\hline
+ & Increments the value at the current position that the machine points to. \\ \hline
- & Decrements the value at the current position that the machine points to. \\ \hline
\textless & Moves the pointer one cell to the left.                                  \\ \hline
\textgreater & Moves the pointer one cell to the right.                                 \\ \hline
[ & Jumps after the corresponding closed bracket when the value at the current cell is 0. \\ \hline
] & Jumps after the corresponding open bracket when the value at the current cell is not 0.   \\ \hline
. & Outputs the value at the current cell. \\ \hline
, & Read a value to be placed at the current cell. \\ \hline
\end{tabularx}
\caption{Brainfuck commands and their descriptions.\cite{BFWiki}}
\end{table}

\par The size and number of memory cells in the language are not strictly defined and are typically left to the discretion of the programmer. In the context of this implementation, a memory model consisting of \textit{65,536} cells was adopted (corresponding to the addressable range of a 16-bit pointer) with each cell represented as an unsigned 8-bit value.

\subsection{Programming in the language}
\label{sec:ch2sec1sub1}

\par To facilitate a practical understanding of the Brainfuck language and its idiomatic constructs, let's examine a well known instruction sequence that outputs the string \textit{'Hello World!'} to the screen:

\begin{minted}[
    linenos,                % add line numbers
    fontfamily=tt,          % typewriter font
    fontsize=\small,        % size
    breaklines,             % allow line breaks
    tabsize=4               % size of tab
]{brainfuck}
>++++++++[<+++++++++>-]<.
>++++[<+++++++>-]<+.
+++++++..+++.
>>++++++[<+++++++>-]<++.
------------.
>++++++[<+++++++++>-]<+.
<.
+++.
------.
--------.
>>>++++[<++++++++>-]<+.
\end{minted}

\par For clarity, the code has been formatted so that each line corresponds to the generation of a single character in the output. This organization aids in demonstrating how Brainfuck instructions directly map to output generation.

\par Consider the first line:

\begin{minted}[
    linenos,                % add line numbers
    fontfamily=tt,          % typewriter font
    fontsize=\small,        % size
    breaklines,             % allow line breaks
    tabsize=4               % size of tab
]{brainfuck}
>++++++++[<+++++++++>-]<.
\end{minted}

\par This sequence first increments the second memory cell (cell \#1) to the value 8. It then enters a loop that decrements cell \#1 on each iteration and simultaneously increments the third cell (cell \#2) by 9 per iteration. Once cell \#1 reaches zero, the loop terminates, and cell \#2 holds the value 72, corresponding to the ASCII code for 'H'. The final instruction outputs this value.

\par Similar patterns are applied across subsequent lines to construct the remaining characters of the phrase. Characters such as \textit{'l'}, which appear multiple times, are output using consecutive \texttt{.} commands. The \textit{'o'} from \textit{"Hello"} is reused in \textit{"World"}, demonstrating efficient reuse of computed values.

\par This example illustrates common patterns and idioms in Brainfuck programming. One foundational construct is the multiplication loop:

\begin{minted}[
    linenos,                % add line numbers
    fontfamily=tt,          % typewriter font
    fontsize=\small,        % size
    breaklines,             % allow line breaks
    tabsize=4               % size of tab
]{brainfuck}
(any number of +)[(movement)(any number of +)(reverse movement)-]
\end{minted}

\par This structure multiplies the initial value of a cell by the number of additions performed inside the loop, effectively replicating multiplication via repeated addition.

\par Another fundamental idiom is the cell-clearing pattern, which resets the current cell's value to zero:

\begin{minted}[
    linenos,                % add line numbers
    fontfamily=tt,          % typewriter font
    fontsize=\small,        % size
    breaklines,             % allow line breaks
    tabsize=4               % size of tab
]{brainfuck}
[(- or +)]
\end{minted}

\par Additionally, a commonly used construct involves transferring the value of one cell to another:

\begin{minted}[
    linenos,                % add line numbers
    fontfamily=tt,          % typewriter font
    fontsize=\small,        % size
    breaklines,             % allow line breaks
    tabsize=4               % size of tab
]{brainfuck}
[(pointer movement)+(reverse pointer movement)-]
\end{minted}

\par This operation assumes that the destination cell is initially zero. If this is not the case, the original value will be added rather than overwritten. To enforce a pure move, the destination cell should be cleared first using the aforementioned clearing pattern.

\par These patterns provide valuable insights into the low-level logic of Brainfuck programs and are instrumental in designing optimized interpreters.

\section{Emulator implementation}
\label{sec:ch2sec2}

\par Because of the language's simplicity, there have been many emulators made for it in all kinds of programming languages. As such, the focus will be put on the techniques utilised in the creation of optimized emulators rather than the emulators themselves, as they have been a rather exhausted subject.

\subsection{Simple interpreter}
\label{sec:ch2sec2sub1}

\par A basic Brainfuck interpreter functions by sequentially parsing each character of the input program and executing the corresponding operation on a simulated memory model.

\par This model typically comprises a data array, initialized to zero, alongside a data pointer that tracks the currently active cell. As the interpreter reads the source code, it maps each command to a specific operation: incrementing or decrementing the cell's value (+ or -), shifting the data pointer left or right (< or >), reading a byte of input (,), or outputting the current cell's value (.).

\par Control flow in Brainfuck is handled through loop constructs denoted by the [ and ] characters. When encountering a [ symbol, the interpreter checks the value of the current cell. If the value is zero, execution jumps forward to the instruction immediately following the corresponding closing bracket ]. Conversely, when a ] is reached and the current cell is non-zero, execution jumps backward to the matching opening bracket [.

\par To enable efficient execution of these control flow constructs, the interpreter usually performs a preprocessing step that generates a mapping between matching bracket pairs, allowing for constant-time jumps during interpretation.

\par The entire execution logic is structured as a large \texttt{switch}-based dispatch mechanism, which simplifies the implementation and provides a modular structure that facilitates later extensions, particularly in the context of optimization.

\par Preliminary performance evaluations for this execution model indicate the following average times after \textit{5} runs:

\begin{longtable}{|l|c|c|}
\hline
\textbf{Program} & \textbf{Instruction Count} & \textbf{Execution Time (ms)} \\
\hline
\endfirsthead

\hline
\textbf{Program} & \textbf{Instruction Count} & \textbf{Execution Time (ms)} \\
\hline
\endhead

\hline
\endfoot

\hline
\endlastfoot

mandlebrot.b  & 11452  & 121035        \\
hanoi.b       & 53885  & 92000         \\
alphabet.b    & 186    & 8000          \\
squares.b     & 197    & 18.7          \\
sierpinski.b  & 125    & 3.7           \\
\end{longtable}

\par Higher times have been rounded as the error is small enough to be negligible.

\subsection{Static compilation}
\label{sec:ch2sec2sub2}

\subsubsection{Choosing the code emmiting library}
\label{sec:ch2sec2sub2sec1}

\par In selecting a library for runtime machine code generation, several options were evaluated, including \textit{libjit}, \textit{asmjit}, and \textit{dynasm}.

\par While these libraries offer varying levels of functionality and abstraction, they presented challenges in terms of integration and control. Specifically, \textit{libjit} and \textit{asmjit} were found to be non-trivial to use within the context of this project, as \textit{asmjit} was intended for use in C++, not C and \textit{libjit} was proven to be outdated and not supported anymore.

\par Also \textit{dynasm} introduced its own post-processing phase in the compilation of the program and in the compilation of the Brainfuck program as well, the second of which effectively negated custom optimizations performed at the code generation level, as similar transformations were already applied by its backend.

\par Ultimately, \textit{GNU Lightning} was chosen for its simplicity, minimal overhead, and direct mapping between the emitted instructions and the final machine code. Its lightweight nature and relative portability across platforms made it a suitable choice for this project, particularly in this scenario which requires fine-grained control over the emitted code.

\par \textit{GNU Lightning} exposes a platform-agnostic register abstraction layer through a set of symbolic registers that represent underlying physical registers on the target architecture. These include general-purpose registers, denoted as \textit{R0} to \textit{R2} in \texttt{x86}, which are typically used for integer operations and control flow. Additionally, it provides volatile or temporary registers, such as \textit{V0} to \textit{V4}, which are typically used for transient calculations and are not preserved across function calls.

\par For handling floating-point operations, it defines a separate set of registers from \textit{F0} to \textit{F7}. These represent floating-point hardware registers and are intended for operations involving real numbers.

\subsubsection{GNU Lightning implementation}
\label{sec:ch2sec2sub2sec2}

\par Basic operations such as memory incrementation (\texttt{+}), pointer movement (\texttt{>}, \texttt{<}) are directly mapped to \textit{GNU Lightning} instructions, such as add \textit{jit\_addi} which adds an immediate value to a register, or \textit{jit\_ldr\_c / jit\_str\_c} which loads / stores an 8-bit register to an address in memory.

\par I/O functions (\texttt{.} and \texttt{,}) are mapped to a function call of their respective pointers from the state, and for input, the returned result is stored in \textit{JIT\_R0} which is then stored in memory.

\par Control flow constructs (\texttt{[}, \texttt{]}) are handled using label stacks and conditional branches, with jumps managed via preallocated loop label arrays.

\par Register usage is explicit, relying on general-purpose virtual registers (\textit{JIT\_R0} and \textit{JIT\_R1}) and pointer management through \textit{JIT\_V0}, which holds the memory base.

\par The JIT state is finalized with a prolog and epilog, and the emitted function pointer is stored for later execution. 

\begin{longtable}{|l|c|c|c|}
\hline
\textbf{Program} & \textbf{Instruction Count} & \textbf{JIT Time (ms)} & \textbf{Time \% of interpreter} \\
\hline
\endfirsthead

\hline
\textbf{Program} & \textbf{Instruction Count} & \textbf{JIT Time (ms)} & \textbf{Time \% of interpreter} \\
\hline
\endhead

\hline
\endfoot

\hline
\endlastfoot

mandlebrot.b  & 11452 & 3800          & 3.13\% \\
hanoi.b       & 53885 & 4830          & 5.25\% \\
alphabet.b    & 186   & 745           & 9.31\% \\
squares.b     & 197   & 0.99          & 5.29\% \\
sierpinski.b  & 125   & 0.25          & 6.75\% \\
\end{longtable}

\par A big reduction in execution time can be already observed, but there's still a lot of room for improvements to be done.

\section{Applying optimizations}
\label{sec:ch2sec3}

\subsection{Precalculating jumps}
\label{subsec:ch2sec3sec1}

\par In the Brainfuck language, control flow is implemented using the loop constructs \texttt{[} and \texttt{]}, which form conditional jump instructions based on the value at the current memory cell. A naive implementation might process these instructions dynamically by scanning forward or backward through the code to locate the corresponding matching bracket whenever a jump is needed.

\par At runtime, if the condition is met, the interpreter may search forward to find the matching \texttt{]}, or backward to locate the corresponding \texttt{[}.

\par While functionally correct, this approach is highly inefficient, especially in programs with nested or frequent loops, as it introduces linear-time overhead on each loop entry and exit.

\subsubsection*{Precomputed Jump Table Optimization}

\par To mitigate this inefficiency, jump destinations can be precalculated during the parsing phase of the program. This transforms dynamic control flow resolution into a constant-time operation, enabling the use of a more efficient intermediate representation.

\par The revised instruction format introduces a single unified jump operation:

\begin{itemize}
    \item \textbf{JMP \textit{x}}: Performs a conditional jump of \textit{x} instructions relative to the current position in the code.
    \begin{itemize}
        \item If \textit{x} is \textbf{positive}, the instruction corresponds to a \texttt{[} and is executed \textbf{only if} the value at the current memory cell is \textbf{zero}.
        \item If \textit{x} is \textbf{negative}, the instruction corresponds to a \texttt{]} and is executed \textbf{only if} the value at the current memory cell is \textbf{non-zero}.
    \end{itemize}
\end{itemize}

\par This structure allows for efficient execution by directly referencing the jump destination, avoiding the need for runtime matching of bracket pairs.

\subsection{Instruction folding}
\label{subsec:ch2sec3sec2}

\par Another fundamental and immediate optimization in Brainfuck program analysis involves the elimination of redundant sequences of repeated instructions as Brainfuck source code often includes consecutive repetitions of basic commands, such as:

\begin{minted}[
    linenos,                % add line numbers
    fontfamily=tt,          % typewriter font
    fontsize=\small,        % size
    breaklines,             % allow line breaks
    tabsize=4               % size of tab
]{brainfuck}
+++++++++   // Increment the current cell by 9
>>>>>>      // Move the pointer 6 cells to the right
\end{minted}

\par These repetitive patterns can be consolidated to improve both interpretive efficiency and give more performant emmited code by the JIT compiler.

\subsection{Sequence matching for common patterns}
\label{subsec:ch2sec3sec3}

\subsubsection*{Transformation to Intermediate Representation}

To address this redundancy, a simplified intermediate set instruction is introduced. It aggregates repeated operations into parameterized instructions, more specifically, the following transformations are applied:

\begin{itemize}
    \item \textbf{ADD \textit{x}}: Replaces sequences of \texttt{+} and \texttt{-} instructions. The signed integer \textit{x} indicates the net change to the value at the current cell.
    \begin{itemize}
        \item Example: \texttt{+++++} becomes \texttt{ADD 5}, and \texttt{---} becomes \texttt{ADD -3}.
    \end{itemize}

    \item \textbf{MOV \textit{x}}: Replaces sequences of \texttt{>} and \texttt{<} instructions. The signed integer \textit{x} represents the net change to the data pointer position.
    \begin{itemize}
        \item Example: \texttt{>>>} becomes \texttt{MOV 3}, and \texttt{<<} becomes \texttt{MOV -2}.
    \end{itemize}
\end{itemize}

\par These transformations serve two primary purposes:

\begin{enumerate}
    \item \textbf{Performance Enhancement}: Interpreters and compilers can process compact instructions more efficiently than long sequences of primitive commands.
    \item \textbf{Readability and Analysis}: The intermediate representation is more succinct and expressive, facilitating further optimization and program analysis.
\end{enumerate}

\subsubsection*{Regex-like Pattern Recognition in Parsing}

\par The code parser operates in a stack-like fashion:

\begin{enumerate}
    \item Read the current instruction.
    \item Determine its type.
    \item If it differs from the top of the stack, push it as a new entry.
    \item If it matches the type, attempt to merge or extend the existing instruction.
\end{enumerate}

\par This parsing mechanism can be extended to support loop pattern recognition: after each new instruction is added to the stack, the parser inspects the top of the stack for known patterns (such as \texttt{3j xa -1j}).

\par If a match is found, the sequence is replaced with the appropriate high-level instruction, such as \texttt{CLR}.
\par Consider the Brainfuck loop:

\begin{verbatim}
[-]
\end{verbatim}

\par Semantically, this loop is equivalent to setting the current memory cell to zero. It repeatedly decrements the cell until it reaches zero, and the loop exits. Despite its simplicity, the execution time of this loop is proportional to the value of the cell, which may range up to 255 (in the case of 8-bit cells). Thus, the runtime cost of this idiom can be substantial.

\subsubsection*{Replacing Common Loops with Custom Instructions}

\par To address this inefficiency, a specialized instruction can be introduced into our intermediate representation:

\begin{itemize}
    \item \textbf{CLR}: Sets the value at the current memory cell to zero.
\end{itemize}

\par This transformation avoids executing multiple decrement operations and conditional jumps, replacing them with a single deterministic operation.

\subsubsection*{Identifying Loops Worth Optimizing}

\par Not all loops merit optimization; hence, we employ a heuristic based on dynamic profiling to identify high-frequency loops.

\par During an initial instrumentation pass of the program, we count the number of executions of each top-level loop. Each loop is recorded with the following format:

\par For example, in the \texttt{mandlebrot.b} program, the following raw data can be observed (\texttt{j} - jump, \texttt{a} - add, \texttt{m} - move):

\begin{verbatim}
number of executions | sequence
157090277 6j -1a 9m 1a -9m -4j
46993495  3j -1a -1j
25555337  6j -1a -2m 1a 2m -4j
...
\end{verbatim}

\par Structural patterns can now be easier to observe, for instance, the pattern:

\begin{verbatim}
3j xa -1j
\end{verbatim}

\par (where \texttt{x} is any non-zero constant) represents any loop of the form:

\begin{verbatim}
[-]  [+]  [++++]  [---]
\end{verbatim}

\par Each of these is semantically equivalent to clearing the current cell. As such, they can be replaced by the custom \texttt{CLR} instruction in the intermediate representation.

\begin{longtable}{|l|c|c|c|c|c|}
\hline
\textbf{Program} & \textbf{Size} & \textbf{Interpreter (ms)} & \textbf{Interp as JIT\%} & \textbf{\% of original} \\
\hline
\endfirsthead
\hline
\textbf{Program} & \textbf{Size} & \textbf{Interpreter (ms)} & \textbf{JIT as Interp\%} & \textbf{\% of original} \\
\hline
\endhead
\hline
\endfoot
mandlebrot.b & 2915  & 14390 & 1326.26\% & 11.89\% \\
alphabet.b   & 80    & 1177  & 3138.66\% & 14.72\% \\
hanoi.b      & 9830  & 1010  & 2927.55\% & 1.10\% \\
\hline
\end{longtable}

\begin{longtable}{|l|c|c|c|c|c|}
\hline
\textbf{Program} & \textbf{Size} & \textbf{JIT (ms)} & \textbf{JIT as Interp\%} & \textbf{\% of original} \\
\hline
\endfirsthead
\hline
\textbf{Program} & \textbf{Size} & \textbf{JIT (ms)} & \textbf{JIT as Interp\%} & \textbf{\% of original} \\
\hline
\endhead
\hline
\endfoot
mandlebrot.b & 2915  & 1085   & 7.53\%  & 28.55\% \\
alphabet.b   & 80    & 37.5   & 3.19\%  & 5.03\% \\
hanoi.b      & 9830  & 34.5   & 3.42\%  & 0.715\% \\
\hline
\end{longtable}

\par These results demonstrate that the JIT offers a significant performance advantage over the interpreter, primarily due to the elimination of instruction fetching and decoding overhead and that the applied optimizations have a substantial impact on the execution time across the tested programs.

\section{Testing}
\label{sec:ch2sec4}

\par The test runner initializes the testing sequence by invoking a series of focused unit tests: one that verifies the correct setup of the memory state, one that checks the fidelity of source code parsing and transformation, and others that validate the interpreter's initialization and execution logic.

\par For builds configured with \textit{GNU Lightning} support, the test suite conditionally includes one that tests the initialization of the JIT backend and one that verifies the execution of JIT-compiled code.