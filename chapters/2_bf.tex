\chapter{Brainfuck}
\label{chap:ch2}

\par Brainfuck, or as it is usually shortened for academical purposes, BF, is an esoteric programming language designed to be minimalistic and intentionally difficult to program in.

\par It was created in 1993 by Swiss student Urban Müller, primarily as a challenge for programmers.

\par Despite its simplicity, Brainfuck is Turing-complete, meaning it has the computational power to perform any calculation that can be computed by a modern computer, given sufficient resources.

\par The language operates on an array of memory cells, each initially set to zero, and uses a series of eight commands to manipulate the data and control program flow.

\par Brainfuck's syntax consists of only eight commands, these being instructions to move the data pointer, modify the value of the current memory cell, input or output data, and control loops.

\par The language's simplicity—combined with its deliberately difficult syntax—forces programmers to think creatively about how to implement basic operations like arithmetic or data manipulation.

\par As a result, Brainfuck is more often used in programming challenges and competitions rather than for practical software development.

\par The language has been made intentionally difficult to use for practical application, which has made it a popular subject of study within the community of esoteric programming languages (esolangs).

\par Brainfuck often serves as an example of the power of minimalism in programming, demonstrating that even with extremely limited resources, a fully functional computational system can be constructed.

\par It challenges conventional programming paradigms, as the programmer is forced to manage memory and control flow manually, akin to working in assembly or machine code.

\clearpage

\section{Machine specification}
\label{sec:ch2sec1}

\begin{table}[H]
\centering
\begin{tabularx}{\textwidth}{| c | Y |}
\hline
+ & Increments the value at the current position that the machine points to. \\ \hline
- & Decrements the value at the current position that the machine points to. \\ \hline
\textless & Moves the pointer one cell to the left.                                  \\ \hline
\textgreater & Moves the pointer one cell to the right.                                 \\ \hline
[ & Jumps after the corresponding closed bracket when the value at the current cell is 0. \\ \hline
] & Jumps after the corresponding open bracket when the value at the current cell is not 0.   \\ \hline
. & Outputs the value at the current cell. \\ \hline
, & Read a value to be placed at the current cell. \\ \hline
\end{tabularx}
\caption{BF commands and their descriptions}
\end{table}

\clearpage

\subsection{Programming in the language}
\label{sec:ch2sec1sub1}

\par For us to properly understand how to use the language in a useful manner, let's take a look at a sample used for printing 'Hello World!' to the screen:
\begin{minted}[
    linenos,                % add line numbers
    fontfamily=tt,          % typewriter font
    fontsize=\small,        % size
    breaklines,             % allow line breaks
    tabsize=4               % size of tab
]{brainfuck}
>++++++++[<+++++++++>-]<.
>++++[<+++++++>-]<+.
+++++++..+++.
>>++++++[<+++++++>-]<++.
------------.
>++++++[<+++++++++>-]<+.
<.
+++.
------.
--------.
>>>++++[<++++++++>-]<+.
\end{minted}

\par I took the liberty of splitting the code by each character that is outputted to the screen such that it can be properly observed how the language works.

\par Let's take a look at the code, line by line:

\begin{minted}[
    linenos,                % add line numbers
    fontfamily=tt,          % typewriter font
    fontsize=\small,        % size
    breaklines,             % allow line breaks
    tabsize=4               % size of tab
]{brainfuck}
>++++++++[<+++++++++>-]<.
\end{minted}

\par This instruction sequence sets the value of the cell at index \#1 to 8 and in a continous loop decrements that value and increments cell \#2 nine times for each decrement. Thus it sets cell \#2 to value 72, ASCII for the character 'H'.

\par Similar steps are done for all of the other characters except the ones that repeat multiple times, like 'l' which is printed twice by using two dot instructions in the third line, or reusing the value of 'o' from 'Hello' in 'World'.

\par This sequence constitutes a good introduction to the matter at hand as, by coding in the language, common programming patterns can be observed such as the multiplication pattern:

\begin{minted}[
    linenos,                % add line numbers
    fontfamily=tt,          % typewriter font
    fontsize=\small,        % size
    breaklines,             % allow line breaks
    tabsize=4               % size of tab
]{brainfuck}
(any number of pluses)[(direction change)(any number of pluses)(reversed direction change)-]
\end{minted}

\par There are also other commonly used patterns, such as:

\begin{minted}[
    linenos,                % add line numbers
    fontfamily=tt,          % typewriter font
    fontsize=\small,        % size
    breaklines,             % allow line breaks
    tabsize=4               % size of tab
]{brainfuck}
The cell clearing pattern, that sets a cell to 0:
[-]
\end{minted}

\begin{minted}[
    linenos,                % add line numbers
    fontfamily=tt,          % typewriter font
    fontsize=\small,        % size
    breaklines,             % allow line breaks
    tabsize=4               % size of tab
]{brainfuck}
The cell moving pattern, that moves a value to a cell relative to it:
[(any number of moves)+(same amount of reversed moves)-]
To be noted that this assumes initial cell value 0, if that value is not 0 it will be added instead.
If you need it to be specifically moved, you can combine the previous clear pattern with this one.
\end{minted}

\clearpage

\section{Emulator implementation}
\label{sec:ch2sec2}

\par Because of the language's simplicity, there have been many emulators made for it in all kinds of programming languages. As such, the focus will be put on the techniques utilised in the creation of optimized emulators rather than the emulators themselves, as they have been a rather exhausted subject.


\subsection{Emulating yourself?}
\label{sec:ch2sec2sub1}

\par 

\subsection{Simple interpreter}
\label{sec:ch2sec2sub2}

\par Lorem ipsum dolor sit amet, consectetur adipiscing elit, sed do eiusmod tempor incididunt ut labore et dolore magna aliqua. Ut enim ad minim veniam, quis nostrud exercitation ullamco laboris nisi ut aliquip ex ea commodo consequat. Duis aute irure dolor in reprehenderit in voluptate velit esse cillum dolore eu fugiat nulla pariatur. Excepteur sint occaecat cupidatat non proident, sunt in culpa qui officia deserunt mollit anim id est laborum

\subsection{Static compilation}
\label{sec:ch2sec2sub3}

\par Lorem ipsum dolor sit amet, consectetur adipiscing elit, sed do eiusmod tempor incididunt ut labore et dolore magna aliqua. Ut enim ad minim veniam, quis nostrud exercitation ullamco laboris nisi ut aliquip ex ea commodo consequat. Duis aute irure dolor in reprehenderit in voluptate velit esse cillum dolore eu fugiat nulla pariatur. Excepteur sint occaecat cupidatat non proident, sunt in culpa qui officia deserunt mollit anim id est laborum

\section{Applying optimizations}
\label{sec:ch2sec3}

\par Lorem ipsum dolor sit amet, consectetur adipiscing elit, sed do eiusmod tempor incididunt ut labore et dolore magna aliqua. Ut enim ad minim veniam, quis nostrud exercitation ullamco laboris nisi ut aliquip ex ea commodo consequat. Duis aute irure dolor in reprehenderit in voluptate velit esse cillum dolore eu fugiat nulla pariatur. Excepteur sint occaecat cupidatat non proident, sunt in culpa qui officia deserunt mollit anim id est laborum

\subsection{Precalculating jumps}
\label{subsec:ch2sec3sec1}

\par Lorem ipsum dolor sit amet, consectetur adipiscing elit, sed do eiusmod tempor incididunt ut labore et dolore magna aliqua. Ut enim ad minim veniam, quis nostrud exercitation ullamco laboris nisi ut aliquip ex ea commodo consequat. Duis aute irure dolor in reprehenderit in voluptate velit esse cillum dolore eu fugiat nulla pariatur. Excepteur sint occaecat cupidatat non proident, sunt in culpa qui officia deserunt mollit anim id est laborum

\subsection{Instruction folding}
\label{subsec:ch2sec3sec2}

\par Lorem ipsum dolor sit amet, consectetur adipiscing elit, sed do eiusmod tempor incididunt ut labore et dolore magna aliqua. Ut enim ad minim veniam, quis nostrud exercitation ullamco laboris nisi ut aliquip ex ea commodo consequat. Duis aute irure dolor in reprehenderit in voluptate velit esse cillum dolore eu fugiat nulla pariatur. Excepteur sint occaecat cupidatat non proident, sunt in culpa qui officia deserunt mollit anim id est laborum

\subsection{Sequence matching for common patterns}
\label{subsec:ch2sec3sec3}

\par Lorem ipsum dolor sit amet, consectetur adipiscing elit, sed do eiusmod tempor incididunt ut labore et dolore magna aliqua. Ut enim ad minim veniam, quis nostrud exercitation ullamco laboris nisi ut aliquip ex ea commodo consequat. Duis aute irure dolor in reprehenderit in voluptate velit esse cillum dolore eu fugiat nulla pariatur. Excepteur sint occaecat cupidatat non proident, sunt in culpa qui officia deserunt mollit anim id est laborum

\section{Testing}
\label{sec:ch2sec4}

\par 