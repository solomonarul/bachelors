\chapter{Introduction}
\label{chap:ch1}

\section*{About}
\label{chap:ch1sec1}

\par Emulators play a vital role in computing by enabling software to run on hardware or virtual platforms distinct from their original execution environments.

\par Their applications span a wide range of domains, including legacy video game preservation, processor simulation, the development of sandboxed runtime systems, and more.

\par Two fundamental techniques underpin most emulator implementations: interpretation and just-in-time (JIT) compilation.

\par A clear understanding of the strengths and limitations of these approaches is crucial for those engaging in low-level systems development or virtual machine construction.

\par This thesis offers an introductory exploration of interpretation and JIT compilation, supported by practical implementation and performance analysis.

\par To illustrate these techniques in a manageable and pedagogically valuable context, the study focuses on two minimal programming languages: Brainfuck and CHIP-8.

\par Brainfuck, a minimalist yet Turing-complete esoteric language, and CHIP-8, a simple virtual machine historically used for teaching early computing and game development concepts, serve as effective case studies due to their simplicity and well-defined behavior.

\par The primary objectives of this thesis are as follows:

\begin{itemize}
	\item To implement both interpreters and JIT compilers for Brainfuck and CHIP-8.
	\item To explore and apply basic runtime optimizations.
	\item To compare the performance and complexity of each technique using a set of representative benchmark programs.
	\item To explore application design by creating a proper modularised architecture whith unit and / or integration tests.
\end{itemize}

\clearpage

\par The organization of the thesis is outlined as follows:

\begin{itemize}
    \item \textbf{Chapter 2} examines the Brainfuck programming language, focusing on its implementation through both interpretation and just-in-time (JIT) compilation, along with a comparative analysis of their respective performance results.
    \item \textbf{Chapter 3} explores the historical evolution and variations of the CHIP-8 virtual machine, detailing the implementation strategies adopted.
    \item \textbf{Chapter 4} provides an in-depth analysis of the modular architecture of the main application, describing the integration and interaction of its components and the rationale behind key design decisions.
    \item \textbf{Chapter 5} summarizes the core findings of the study, discusses its limitations, and proposes potential directions for future research and development in the field of software emulation.
\end{itemize}

\section*{Related work}
\label{chap:ch1sec2}

\par The programming languages analyzed in this study, have seen limited engagement in academic literature, likely due to Brainfuck's intentionally provocative nomenclature and CHIP-8's roots in hobbyist computing, far from academia.

\par Nevertheless, both languages have found specific applications in research and education. CHIP-8 has been employed as an instructional tool for teaching computer architecture and virtual machine concepts in undergraduate curricula, demonstrating its relevance in contemporary pedagogy despite its retro origins \cite{Chip8Applications2019}.

\par Brainfuck, on the other hand, has appeared in more niche academic contexts. It has been discussed in the broader framework of esoteric and conceptual programming languages \cite{BFEsolang2015, BFConceptual2017}, explored in parallel hardware design implementations \cite{BFHardware2016}, used in preliminary studies on self-interpreting languages \cite{BFSelfInterpreter2003}, and even utilized as a domain for reinforcement learning in program synthesis and compiler fuzzing research \cite{BFReinforcementLearining2022}.

\par These contributions, while scattered, highlight the potential of these languages as compact and expressive models for exploring core principles in interpretation, compilation, and system design.